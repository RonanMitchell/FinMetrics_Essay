\documentclass[11pt,preprint, authoryear]{elsarticle}

\usepackage{lmodern}
%%%% My spacing
\usepackage{setspace}
\setstretch{1.2}
\DeclareMathSizes{12}{14}{10}{10}

% Wrap around which gives all figures included the [H] command, or places it "here". This can be tedious to code in Rmarkdown.
\usepackage{float}
\let\origfigure\figure
\let\endorigfigure\endfigure
\renewenvironment{figure}[1][2] {
    \expandafter\origfigure\expandafter[H]
} {
    \endorigfigure
}

\let\origtable\table
\let\endorigtable\endtable
\renewenvironment{table}[1][2] {
    \expandafter\origtable\expandafter[H]
} {
    \endorigtable
}


\usepackage{ifxetex,ifluatex}
\usepackage{fixltx2e} % provides \textsubscript
\ifnum 0\ifxetex 1\fi\ifluatex 1\fi=0 % if pdftex
  \usepackage[T1]{fontenc}
  \usepackage[utf8]{inputenc}
\else % if luatex or xelatex
  \ifxetex
    \usepackage{mathspec}
    \usepackage{xltxtra,xunicode}
  \else
    \usepackage{fontspec}
  \fi
  \defaultfontfeatures{Mapping=tex-text,Scale=MatchLowercase}
  \newcommand{\euro}{€}
\fi

\usepackage{amssymb, amsmath, amsthm, amsfonts}

\def\bibsection{\section*{References}} %%% Make "References" appear before bibliography


\usepackage[round]{natbib}

\usepackage{longtable}
\usepackage[margin=2.3cm,bottom=2cm,top=2.5cm, includefoot]{geometry}
\usepackage{fancyhdr}
\usepackage[bottom, hang, flushmargin]{footmisc}
\usepackage{graphicx}
\numberwithin{equation}{section}
\numberwithin{figure}{section}
\numberwithin{table}{section}
\setlength{\parindent}{0cm}
\setlength{\parskip}{1.3ex plus 0.5ex minus 0.3ex}
\usepackage{textcomp}
\renewcommand{\headrulewidth}{0.2pt}
\renewcommand{\footrulewidth}{0.3pt}

\usepackage{array}
\newcolumntype{x}[1]{>{\centering\arraybackslash\hspace{0pt}}p{#1}}

%%%%  Remove the "preprint submitted to" part. Don't worry about this either, it just looks better without it:
\makeatletter
\def\ps@pprintTitle{%
  \let\@oddhead\@empty
  \let\@evenhead\@empty
  \let\@oddfoot\@empty
  \let\@evenfoot\@oddfoot
}
\makeatother

 \def\tightlist{} % This allows for subbullets!

\usepackage{hyperref}
\hypersetup{breaklinks=true,
            bookmarks=true,
            colorlinks=true,
            citecolor=blue,
            urlcolor=blue,
            linkcolor=blue,
            pdfborder={0 0 0}}


% The following packages allow huxtable to work:
\usepackage{siunitx}
\usepackage{multirow}
\usepackage{hhline}
\usepackage{calc}
\usepackage{tabularx}
\usepackage{booktabs}
\usepackage{caption}


\newenvironment{columns}[1][]{}{}

\newenvironment{column}[1]{\begin{minipage}{#1}\ignorespaces}{%
\end{minipage}
\ifhmode\unskip\fi
\aftergroup\useignorespacesandallpars}

\def\useignorespacesandallpars#1\ignorespaces\fi{%
#1\fi\ignorespacesandallpars}

\makeatletter
\def\ignorespacesandallpars{%
  \@ifnextchar\par
    {\expandafter\ignorespacesandallpars\@gobble}%
    {}%
}
\makeatother

\newenvironment{CSLReferences}[2]{%
}

\urlstyle{same}  % don't use monospace font for urls
\setlength{\parindent}{0pt}
\setlength{\parskip}{6pt plus 2pt minus 1pt}
\setlength{\emergencystretch}{3em}  % prevent overfull lines
\setcounter{secnumdepth}{5}

%%% Use protect on footnotes to avoid problems with footnotes in titles
\let\rmarkdownfootnote\footnote%
\def\footnote{\protect\rmarkdownfootnote}
\IfFileExists{upquote.sty}{\usepackage{upquote}}{}

%%% Include extra packages specified by user

%%% Hard setting column skips for reports - this ensures greater consistency and control over the length settings in the document.
%% page layout
%% paragraphs
\setlength{\baselineskip}{12pt plus 0pt minus 0pt}
\setlength{\parskip}{12pt plus 0pt minus 0pt}
\setlength{\parindent}{0pt plus 0pt minus 0pt}
%% floats
\setlength{\floatsep}{12pt plus 0 pt minus 0pt}
\setlength{\textfloatsep}{20pt plus 0pt minus 0pt}
\setlength{\intextsep}{14pt plus 0pt minus 0pt}
\setlength{\dbltextfloatsep}{20pt plus 0pt minus 0pt}
\setlength{\dblfloatsep}{14pt plus 0pt minus 0pt}
%% maths
\setlength{\abovedisplayskip}{12pt plus 0pt minus 0pt}
\setlength{\belowdisplayskip}{12pt plus 0pt minus 0pt}
%% lists
\setlength{\topsep}{10pt plus 0pt minus 0pt}
\setlength{\partopsep}{3pt plus 0pt minus 0pt}
\setlength{\itemsep}{5pt plus 0pt minus 0pt}
\setlength{\labelsep}{8mm plus 0mm minus 0mm}
\setlength{\parsep}{\the\parskip}
\setlength{\listparindent}{\the\parindent}
%% verbatim
\setlength{\fboxsep}{5pt plus 0pt minus 0pt}



\begin{document}



\begin{frontmatter}  %

\title{Government Bond Yield Spreads as a Measure of Financial Risk}

% Set to FALSE if wanting to remove title (for submission)




\author[Add1]{Ronan Morris}
\ead{22876634}





\address[Add1]{Stellenbosch University}



\vspace{1cm}





\vspace{0.5cm}

\end{frontmatter}

\setcounter{footnote}{0}



%________________________
% Header and Footers
%%%%%%%%%%%%%%%%%%%%%%%%%%%%%%%%%
\pagestyle{fancy}
\chead{}
\rhead{}
\lfoot{}
\rfoot{\footnotesize Page \thepage}
\lhead{}
%\rfoot{\footnotesize Page \thepage } % "e.g. Page 2"
\cfoot{}

%\setlength\headheight{30pt}
%%%%%%%%%%%%%%%%%%%%%%%%%%%%%%%%%
%________________________

\headsep 35pt % So that header does not go over title




\hypertarget{introduction}{%
\section{Introduction}\label{introduction}}

Viewing economic risk through the lens of the \emph{Yield to Maturity}
(YTM) of government bonds is a common approach in the financial risk
literature. YTM serves as an intuitive measure of macro-economic risk.
The discussion in this paper employs a geo-spatial analysis of select
macro-economic factors, the yield distributions of selected nations, as
well as their government bond yields over time. These descriptive
statistics contextualize the analysis of yield spreads at a later
juncture.

Notably, the geo-spatial analysis reveals that no single macro-economic
factor can be used to perfectly predict bond yields. The examination of
yields over time and yield distributions initiates a discussion relating
to the susceptibility of nations to economic tail events. The yields
over time are evaluated alongside the Volatility Index (VIX), inflation,
and exchange rate volatility. Government bonds in the developed world
exhibit significant yield-trend similarity, which is a contrast to the
erratic trends observed in the yields of BRICS nations and the
developing world.

When assessing yield spreads, using the United States (US) government
bond as a benchmark, a departure from historical norms is evident. The
US emerges as a riskier investment, falling behind Germany and Japan in
risk profile. Furthermore, an alarming trend in South Africa is
observed. The nation exhibits a heightened risk profile since 2010,
further surpassing its BRICS counterparts, such as China - and even
India.

\hypertarget{literature-review}{%
\section{Literature Review}\label{literature-review}}

\hypertarget{the-calculation-of-bond-yield}{%
\subsection{The Calculation of Bond
Yield}\label{the-calculation-of-bond-yield}}

The assessment of bonds as worthwhile assets differs from the evaluation
of risky assets. Financial analysts often use the term ``yield'' in
order to contextualize bond performance. This can take various forms;
Current Yield, Yield-to-Call, Yield-to-Worst, and YTM. The widely
adopted practice in finance involves employing YTM to calculate yield
spreads, as is the case in this paper. In Lawrence \& Shankar (2007:92),
the relationship between a bond price and YTM is calculated as:

\[ P = \frac{C}{(1+i)} + \frac{C}{(1+i)^2} + \ldots + \frac{M}{(1+i)^n} \]

In the above equation, \emph{P} is the price of a bond, \emph{C} is the
coupon rate, \emph{M} is the principal payment, and \emph{i} - which can
be inferred - is the YTM. From this, it can be concluded that there is
an inverse relationship between the price of a bond and the YTM,
\emph{ceteris paribus}. This inverse relationship is a useful tool for
understanding why bond yields are a strong indicator of risk. This is
elaborated on in the subsequent section. It is important to note that a
core assumption of this formula is that the bond buyer will hold until
maturity and reinvest each interest payment at the same rate. Thus, it
is an estimate of what one could expect to earn from purchasing a bond.

\hypertarget{the-economic-implications-of-bond-yields}{%
\subsection{The Economic Implications of Bond
Yields}\label{the-economic-implications-of-bond-yields}}

The evaluation of bond yields involves dissecting them into two
components: a risk-free rate, indicative solely of time preferences, and
a risk premium. The risk premium encompasses any macro-economic risks;
inflation, exchange rate fluctuations, or liquidity constraints (Lo
Conte, 2009:344). A nation exhibiting a higher risk profile requires
that investors are compensated with a higher yield when purchasing
government debt. Therefore, yields are often used as a strong indicator
of economic risk. So much so that Manganelli \& Wolswijk (2009:203)
posit that two assets with identical cash flows should exhibit a yield
spread of zero if they also have identical risk-return characteristics.

When selecting a benchmark bond, government securities are a common
choice (Lo Conte, 2009:341). This paper makes use of US Governments bond
yields as the benchmark when calculating yield spreads. Many analysts
consider US government bonds to be the world's \emph{status-quo} store
of value (He, Krishnamurthy, \& Milbradt, 2016:519). A consistently high
demand for US Treasury Bills has led to low yields historically (He
\emph{et al.}, 2016:519).

In a fixed benchmark scenario, such as a yield spread analysis,
variations in investor risk appetite become obvious. During risk-averse
periods, the price of higher risk bonds will fall (in line with their
value) and the yield will increase. Safer assets will likely see their
value rise, and their yield decrease (Lo Conte, 2009:351). Manganelli \&
Wolswijk (2009:203) claim this time-varying risk aversion is a function
of economic risk, and this is a common interpretation in financial risk
literature.

Unlike the equities market, the appeal of the bond market lies in its
role as a stable investment vehicle for institutional portfolios -
particularly banks and low-risk funds. For example, a substantial
allocation of bonds would be a prominent feature of the ``safe asset
portfolios'' of central banks (He \emph{et al.}, 2016:519). This is why
a high-yield-bond is not as desired as a high-return-stock. Instead, the
price and yield are inversely related.

Some have described the safety of an asset as endogenous; the actions of
investors can make a bond more stable than macro-economic factors
indicate (He \emph{et al.}, 2016:523). For example, Lo Conte (2009:343)
contends that the ratio of total government debt to Gross Domestic
Product (GDP) is a robust measure of economic risk, significantly
influencing EU bond yields, but this fact is not universal.

Between 2006-2016, US government debt surged. This trend was not
reflected in government bond yields (He et al., 2016:519). He \emph{et
al.} (2016:519) argue that this was because investors did not feel as
confident in the economic outlook of any developed nation as they did
about the US. The analysis in this paper argues that investors are
beginning to behave in a manner reflective of this additional risk. This
conclusion is aided by the use of an updated data set spanning
2010-2023.

\hypertarget{descriptive-statistics}{%
\section{Descriptive Statistics}\label{descriptive-statistics}}

This section is dedicated to discussing the interplay between bond
yields and macro-economic indicators. Particular attention is paid
towards inflation, \emph{US-EU} exchange rate volatility, the VIX, GDP
Growth, and Debt to GDP ratios. The juxtaposition of government bond
yields with these indicators will explain the decisions made in the
subsequent spread analysis, such as why the United States is the
benchmark bond.

Commencing the descriptive analysis is a discussion on the significance
of inflation, growth, and debt ratios as economic indicators correlated
with 2-year bond yields. This discussion is facilitated by employing
geo-spatial plots, which exhibit the average rate of each indicator for
individual countries since 2010. While the use of decade-long averages
provides an intuitive perspective, one must acknowledge the time
dynamics associated with macro-economic indicators. Consequently, a more
detailed examination is merited.

This temporal element is addressed by graphically representing bond
yields for developing countries, developed nations, and BRICS (Brazil,
Russia, India, China, South Africa) since 2010. Accompanying these
graphics is the corresponding inflation trend for each nation during the
same period. These plots include blue regions denoting periods of high
VIX, and red zones signifying increased \emph{US-EU} exchange rate
volatility. The choice of 2-year bond yields is crucial, as their
shorter time to maturity means they are sensitive to fluctuations in
short run economic conditions.

Concluding the descriptive analysis, attention is directed towards the
distribution of bond yields for each developed country and each BRICS
nation. Analyzing these distributions is important for determining the
susceptibility of different nations to economic tail events.

\hypertarget{geo-spatial-analysis}{%
\subsection{Geo-Spatial Analysis}\label{geo-spatial-analysis}}

Yield differentials, especially in developing nations, cannot be
entirely explained by discussing debt ratios or inflation rates.
According to Lo Conte (2009:343), financial experts must consider the
ability of a nation to meet debt obligations in U.S. dollars. This can
occur through generating a trade surplus, or using foreign currency
reserves. The geo-spatial analysis in this section emphasizes that no
single macro-economic factor perfectly predicts bond yields.

In Figure 3.1, each country in the data set is assigned a color to
exhibit its ranking in a specific macro-economic factor compared to
other countries. For example, Venezuela has had the highest average bond
yield since 2010, while Germany and Japan tend to have some of the
lowest. Notably, Russia has a very low average debt ratio, but its bond
yield has been among the highest. This seems less connected to
traditional macro-economic factors and more likely linked to
geopolitical risk.

\begin{figure}

{\centering \includegraphics[width=1\linewidth,height=0.5\textheight]{Images/Full_Geo} 

}

\caption{Average Macro-economic Indicators Since 2010 \label{Figure3.1}}\label{fig:unnamed-chunk-2}
\end{figure}

One can argue that inflation holds significance for investors assessing
government bond risk, given the similarity between high average
inflation and high average bond yields. Although not universally
applicable, this trend is displayed in most cases. In contrast, the debt
ratio appears less connected to bond yields. Japan has the world's
highest debt ratio but the lowest bond yield, and Nigeria has a low debt
ratio yet it ranks among the nations with the highest average bond
yields.

This can be attributed to the plots presenting all macro-economic
factors \emph{in levels}. It is likely that if a nation unexpectedly
accumulates more debt than anticipated by investors, this would
adversely impact the perceived risk profile of the country.
Additionally, there is likely a selection bias associated with this
variable; countries that are reliable accumulate more debt and higher
debt ratios, given the expectation of repayment. In contrast, riskier
countries may not be afforded the opportunity to accumulate substantial
debt.

Furthermore, average GDP growth is an inadequate predictor of bond
yields. South Africa exhibits notably low economic growth relative to
other countries in the data set. It also exhibits a high bond yield. If
this was a consistent relationship, one would anticipate high bond
yields from Germany, France, and the UK. However, these countries
exhibit some of the lowest average bond yields in the data set.

\hypertarget{bond-yields}{%
\subsection{Bond Yields}\label{bond-yields}}

In the period spanning 1999 to the onset of the global financial crisis
(GFC), there was a convergence in yield differentials among European
Union (EU) nations. This was largely attributable to the the
implementation of a common currency (Lo Conte, 2009:342). Consequently,
the eradication of exchange rate risk left only credit and liquidity
risk as determinants of yield differentials. Figure 3.2 displays the
modern trend; there is still a similarity in the differentials between
the bond yields of all developed nations, including EU nations. This
trend contrasts with the erratic differentials observed among developing
nations, as displayed in Figure 3.3.

Developed nations exhibit similar trends during periods of crisis,
including the 2013 episode of substantial exchange rate volatility, as
well as elevated VIX observed in 2020. Additionally, a homogeneous
escalation in bond yields is observed during the alarming inflation
witnessed in 2021. A notable observation pertaining to the risk level of
developed nations as a broad group is the confined range of these bond
yields. The range sits between being slightly negative, and 4\%. This
stands in contrast to the varied levels observed in the yields of
developing nations. Furthermore, there exists a co-movement between bond
yields and inflation in the developed world. This co-movement emphasizes
the diminished effect of other economic risk factors on bond yields in
developed economies.

\begin{figure}

{\centering \includegraphics[width=0.9\linewidth,height=0.5\textheight]{Images/Developed_Bond_Yields} 

}

\caption{Bond Yields of Developed Nations \label{Figure3.2}}\label{fig:unnamed-chunk-3}
\end{figure}

In the 1980s, emerging economies had a low presence in the market for
the issuance of debt (3.5 billion dollars in 1989). However, this
changed rapidly, with the levels of debt issued by developing nations
reaching 102 billion dollars by 1996 (Lo Conte, 2009:342). Figure 3.3
lacks the bond yields of Venezuela and Mexico. The exclusion of these
nations is due to decisions by the respective governments that were
perceived negatively by investors. This means that the short-term bond
yields of Venezuela and Mexico require a unique scale to visualize, due
to the considerable impact of the unfavorable investor sentiment
resulting from these actions.

\begin{figure}

{\centering \includegraphics[width=0.9\linewidth,height=0.5\textheight]{Images/Developing_Bonds_Yields} 

}

\caption{Bond Yields of Developing Nations \label{Figure3.3}}\label{fig:unnamed-chunk-4}
\end{figure}

Certain developing nations do exhibit relatively low yields, such as
Bulgaria, where the 2-year bond yield has been negative since 2016.
Although classified as an upper-middle-income country by the World Bank
(World Bank, 2022), Bulgaria has a financial history reflecting a
commitment to meeting debt obligations. Russia, alongside most members
of the BRICS consortium, shares the classification of
``upper-middle-income'', according to the World Bank (World Bank, 2022).
However, Russia's bond yields diverge significantly from that of
Bulgaria. This is unsurprising given Russia's involvement in two
international conflicts within the past decade.

\begin{figure}

{\centering \includegraphics[width=0.9\linewidth,height=0.5\textheight]{Images/BRICS_Bonds_Yields} 

}

\caption{Bond Yields of BRICS Nations \label{Figure3.4}}\label{fig:unnamed-chunk-5}
\end{figure}

BRICS nations lack uniformity in bond yields. Furthermore, the bond
yields of these countries do not demonstrate a strong synchronization
with inflation. While inflation remains a risk factor in BRICS nations,
they also contend with substantial exposure to exchange rate volatility,
geopolitical uncertainties, and liquidity risks. Inflation, in the
overall risk profile of BRICS bond yields, is proportionally less
important than it is in the risk profile of developed nations. There are
many other risks to consider as well.

\hypertarget{yield-distributions}{%
\subsection{Yield Distributions}\label{yield-distributions}}

In Figure 3.5, the credibility of the US government bond as a benchmark
becomes evident. The yields associated with this have symmetrical
distribution, and an insensitivity to tail events. However, the US
government bond is not the least risky among those offered by developed
nations. The figure suggests that this title should be attributed to
bonds issued by Japan or Germany. Japan is notorious for often
struggling with deflation rather than inflation (Ueda, 2012:176), thus
the yields of Japanese government bonds remain below the average yield
of the US alternative, represented by the grey dotted line.

\begin{figure}

{\centering \includegraphics[width=1\linewidth,height=0.47\textheight]{Images/Developed_Distributions} 

}

\caption{Bond Yield Distributions of Developed Nations \label{Figure3.5}}\label{fig:unnamed-chunk-6}
\end{figure}

South African bonds are relatively high-risk investments, even when
juxtaposed with the alternatives offered by other BRICS nations. This is
evident in Figure 3.6, where the distribution of the government bond
yields of Brazil, Russia, India, and China all largely reside to the
left of the average South African bond yield (the dotted line). Russia
maintains susceptibility to tail events. Moreover, the bimodal
distribution observed in Indian bond yields suggests a potential rapid
shift from one average yield to another.

\begin{figure}

{\centering \includegraphics[width=1\linewidth,height=0.47\textheight]{Images/BRICS_Distributions} 

}

\caption{Bond Yield Distributions of BRICS Nations \label{Figure3.6}}\label{fig:unnamed-chunk-7}
\end{figure}

\hypertarget{yield-spreads}{%
\section{Yield Spreads}\label{yield-spreads}}

This section examines both 2-year and 10-year yield spreads. This
decision is guided by the fact that 2-year yields are more responsive to
short-term economic fluctuations, while 10-year spreads reflect changes
in investors' long-term economic sentiments, \emph{ceteris paribus}.
Yields are influenced by various factors strictly beyond the economic
outlook of a nation, although I argue that inflation, interest rates,
debt, and other macro-economic factors are often indicative of the
underlying economic strength of a country.

\hypertarget{developed-nations}{%
\subsection{Developed Nations}\label{developed-nations}}

Figure 4.1 illustrates that over time, the United States (the benchmark
bond at zero) has evolved into a comparatively risky investment within
the context of the developed world. Between 2015 and 2020, all developed
nations exhibited lower short-term bond yields than the United States.
In terms of long-term sentiment, there appears to be a sustained
perception among investors that the economic outlook for the United
States is not as optimistic as before. Despite exhibiting less
volatility, the 10-year spread still signifies a prolonged trend towards
the United States government bond being considered a relatively risky
asset compared to developed world alternatives.

\begin{figure}

{\centering \includegraphics[width=0.9\linewidth,height=0.5\textheight]{Images/Developed_Spreads} 

}

\caption{Developed World Bond Yield Spreads \label{Figure4.1}}\label{fig:unnamed-chunk-8}
\end{figure}

The US government bond continues to exhibit lower risk when compared to
a developing nation like South Africa. In Figure 4.2, three distinct
yield spreads are depicted, maintaining the US as the reference point of
zero, alongside Germany and South Africa. Notably, these two nations
have undergone divergent trajectories in risk profile over the past
decade. Germany has evolved to be consistently safer than the United
States, while South Africa has experienced a persistent increase in risk
level. At the onset of the decade, the spread between the two nations
(relative to the US benchmark) stood only at 5\%, but this has since
doubled.

\begin{figure}

{\centering \includegraphics[width=1\linewidth,height=0.51\textheight]{Images/SA-Germany} 

}

\caption{South Africa vs Germany Yield Spread \label{Figure4.2}}\label{fig:unnamed-chunk-9}
\end{figure}

\hypertarget{brics-nations}{%
\subsection{BRICS Nations}\label{brics-nations}}

The short run sentiment of investors in relation to BRICS nations has
generally been stable, with exceptions during the two instances of
violent conflict between Russia and the Ukraine, leading to a notable
surge in short run yields of Russian government bonds. Conversely, the
long-term outlook appears more disconcerting for South Africa. Since
receiving a ``junk'' bond rating in 2017, the long-term yield has
displayed gradual increases, even when compared to other geopolitical
``anchor states'' within BRICS. Although the current Russian government
bond yield surpasses that of South Africa, the five-year trend preceding
COVID-19 implies a lack of investor confidence in the ability of the
South African economy to recover to pre-GFC levels of economic growth.

\begin{figure}

{\centering \includegraphics[width=0.9\linewidth,height=0.5\textheight]{Images/BRICS_Spreads} 

}

\caption{BRICS Bond Yield Spreads \label{Figure4.3}}\label{fig:unnamed-chunk-10}
\end{figure}

Refer to Figure 4.4 for a visual comparison between specific BRICS
nations. China, while still perceived as riskier than the United States,
has demonstrated a noteworthy degree of stability concerning its yield
in comparison to the US government bond. Despite a marginal surge in the
Chinese yield relative to the US yield during the pandemic, it promptly
reverted to its normal level. In contrast, the trajectory of the South
African government bond continues to diverge; the spike in the South
African 10-year bond yield during COVID-19 was notably more pronounced
than its Chinese counterpart. Moreover, over the decade, the 10-year
bond yields of the two nations have consistently drifted apart.

\begin{figure}

{\centering \includegraphics[width=1\linewidth,height=0.51\textheight]{Images/SA-China} 

}

\caption{SA vs China Yield Spread \label{Figure4.4}}\label{fig:unnamed-chunk-11}
\end{figure}

Perhaps the most informative depiction of waning confidence in the South
African risk profile is evident in Figure 4.5. India and South Africa
are often regarded as examples of countries with growth potential
stifled by being caught in a ``middle income trap'' or grappling with
corruption (Transparency International, 2023). In the initial years of
the decade, the yields of both nations, relative to the US, were
comparable and frequently intersected. However, since 2015, South Africa
has consistently been perceived as a riskier country than India. This
trend is concerning, particularly considering that 10-year yields serve
as indicators of longer-term economic prospects. Investors lack
confidence in the long run South African economy, even in comparison to
a nation that was previously regarded as having a similar risk profile.

\begin{figure}

{\centering \includegraphics[width=1\linewidth,height=0.51\textheight]{Images/SA-India} 

}

\caption{SA vs India Yield Spread \label{Figure4.5}}\label{fig:unnamed-chunk-12}
\end{figure}

\hypertarget{conclusion}{%
\section{Conclusion}\label{conclusion}}

Since 2010, the trajectory of United States government bond yields
challenges its historical status as a safe haven asset. Developed
nations like Japan and Germany, as well as some developing countries
like Bulgaria, exhibit lower risk with negative yield spreads. While the
US remains less risky than most developing nations, certain government
bonds, like that of China, are maintaining or even narrowing their yield
spread relative to the US over time.

South Africa, despite the increased relative yield of US government
bonds, has consistently increased its own government bond yield spread
relative to the US. Over the last few years it holds the position of the
riskiest among BRICS government bonds. This sustained trend highlights
South Africa's precarious investment status, necessitating substantial
reforms to address negative investor sentiments and improve economic
prospects.

\newpage

\hypertarget{bibliography}{%
\section{Bibliography}\label{bibliography}}

CBOE Volatility Index: VIX {[}Online{]}. {[}n.d.{]}. Available:
\url{https://fred.stlouisfed.org/series/VIXCLS} {[}2023, December 10{]}.

Central Government Debt {[}Online{]}. {[}n.d.{]}. Available:
\url{https://www.imf.org/external/datamapper/CG_DEBT_GDP@GDD/CHN/FRA/DEU/ITA/JPN/GBR/USA}
{[}2023, December 28{]}.

Consumer Price Index (2010 = 100) {[}Online{]}. {[}n.d.{]}. Available:
\url{https://data.worldbank.org/indicator/FP.CPI.TOTL?end=2022\&start=2010\&view=chart}
{[}2023, December 10{]}.

Corruption Perceptions Index {[}Online{]}. {[}n.d.{]}. Available:
\url{https://www.transparency.org/en/cpi/2022} {[}2024, January 7{]}.

He, Z., Krishnamurthy, A., \& Milbradt, K. 2016. What Makes US
Government Bonds Safe Assets? \emph{The American Economic Review},
106(5):519-523.

Lawrence, E. R. \& Shankar, S. 2007. A Simple Student-Friendly Approach
to the Mathematics of Bond Prices. \emph{Quarterly Journal of Business
and Economics}, 46(4):91-99.

Lo Conte, R. 2009. Government Bond Yields: A Survey. \emph{Giornale
degli Economisti e Annali di Economia}, 68(3):341-369.

Manganelli, S. \& Wolswijk, G. 2009. What Drives Spreads in the Euro
Area Bond Market? \emph{Economic Policy}, 24(58):191-240.

Ueda, K. 2012. Japan's Deflation and the Bank of Japan's Experience with
Nontraditional Monetary Policy. \emph{Journal of Money, Credit and
Banking}, 44(1):175-190.

U.S. to Euro Spot Exchange Rate {[}Online{]}. {[}n.d.{]}. Available:
\url{https://fred.stlouisfed.org/series/DEXUSEU} {[}2023, December
10{]}.

World Bank Group country classifications by income level for FY24
{[}Online{]}. {[}n.d.{]}. Available:
\url{https://blogs.worldbank.org/opendata/new-world-bank-group-country-classifications-income-level-fy24}
{[}2023, December 31{]}.

\bibliography{Tex/ref}





\end{document}
